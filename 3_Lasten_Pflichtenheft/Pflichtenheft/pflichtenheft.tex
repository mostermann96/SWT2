\documentclass[12pt, a4paper]{article}

\usepackage[utf8]{inputenc}
\usepackage[ngerman]{babel}
\usepackage[T1]{fontenc}

\title{Pflichtenheft}

% Kapitel 1 bis 12 sind von Balzert übernommen, 13 und 14 stammen aus der 
% Aufgabenstellung. Ich habe also erstmal alle Punkte übernommen, wir können
% dann noch schauen, ob alle sinnvoll sind oder ob wir uns nur an Balzert 
% halten.


\begin{document}

\maketitle
\pagebreak
\tableofcontents
\pagebreak

\section{Zielbestimmung}
\subsection{Musskriterien}
\subsection{Wunschkriterien}
\subsection{Abgrenzungskriterien}
\pagebreak

\section{Produkteinsatz}
\subsection{Anwendungsbereiche}
\subsection{Zielgruppen}
\subsection{Betriebsbedingungen}
\pagebreak

\section{Produktübersicht}
\pagebreak

\section{Produktfunktionen}
\pagebreak

\section{Produktdaten}
\pagebreak

\section{Produktleistungen}
\pagebreak

\section{Qualitätsanforderungen}
\pagebreak

\section{Benutzeroberfläche}
\pagebreak

\section{Nichtfunktionale Anforderungen}
\pagebreak

\section{Technische Produktumgebung}
\subsection{Software}
\subsection{Hardware}
\subsection{Orgware}
\subsection{Produkt-Schnittstellen}
\pagebreak

\section{Spezielle Anforderungen an die Entwicklungs-Umgebung}
\subsection{Software}
\subsection{Hardware}
\subsection{Orgware}
\subsection{Entwicklungs-Schnittstellen}
\pagebreak

\section{Gliederung in Teilprodukte}
\pagebreak

\section{Globale Testszenarien/Akzeptanztestfälle}
\pagebreak

\section{Glossar}
\pagebreak

\end{document}
