\documentclass[12pt, a4paper]{article}

\usepackage[utf8]{inputenc}
\usepackage[ngerman]{babel}
\usepackage[T1]{fontenc}
\usepackage{pifont}% http://ctan.org/pkg/pifont
\title{Pflichtenheft}

% Kapitel 1 bis 12 sind von Balzert übernommen, 13 und 14 stammen aus der
% Aufgabenstellung. Ich habe also erstmal alle Punkte übernommen, wir können
% dann noch schauen, ob alle sinnvoll sind oder ob wir uns nur an Balzert
% halten.


\begin{document}

\maketitle
\pagebreak
\tableofcontents
\pagebreak

\section{Zielbestimmung}
\subsection{Musskriterien}
\paragraph{Allgemein}
\begin{itemize}
	\item /M01/ Akteure: Gast, Nutzer, wobei Gast \textless Nutzer (Vererbung)
	\item /M02/ Die Bibliothek soll von Nutzern, also von Lehrstuhlmitarbeitern und Studenten benutzt werden.
	\item /M03/ Es gibt drei Gruppen von Nutzern mit verschiedenen Rechten: Lehrstuhlmitarbeiter, Studenten, Administratoren, wobei Studenten \textless Lehrstuhlmitarbeiter \textless Administratoren (Vererbung)
	\item /M04/ Jeder Nutzer soll zu einer Gruppe gehören
	\item /M05/ Studenten sollen nur zur Gruppe Studenten gehören.
	\item /M06/ Lehrstuhlmitarbeiter sollen entweder zur Gruppe Lehrstuhlmitarbeiter oder Administratoren gehören.
	\item /M07/ Es muss immer mindestens ein Nutzer innerhalb Gruppe Administratoren existieren
	\item /M08/ Eine intuitive, rechnergestützte GUI soll umgesetzt werden
	\item /M09/ Die Produktleistungswerten sollen eingehalten werden
	\item /M10/ Alle Produktdateneinträge sollen gespeichert werden
	\item /M11/ Das System soll von jedem Rechnersystem am Lehrstuhl erreichbar sein
	\item /M12/ Mehrere Nutzer können gleichzeitig auf die Software zugreifen
	\item /M13/ Das System soll Bücher, Zeitschriften und Datenträger aller Art sowie elektronische Medien verwalten können
	\item /M14/ Automatisches Versenden von Mahnungen ist möglich
	\item /M15/ Automatisches Versenden von Rückgabeerinnerungen ist möglich
\end{itemize}
\paragraph{Gast kann ...}
\begin{itemize}
	\item /M16/ eine Suche nach einem Medium über Metadaten ausführen
	\item /M17/ eine Suche nach einem digitalen Medium über den Inhalt ausführen
\end{itemize}
\paragraph{Nutzer kann ...}
\begin{itemize}
	\item /M18/ sich einloggen
	\item /M19/ sich ausloggen
	\item /M20/ physische Medien ausleihen
	\item /M21/ physische Medien vormerken
	\item /M22/ physische Medien verlängern
	\item /M23/ physische Medien zurückgeben
	\item /M24/ digitale Medien ansehen
	\item /M25/ digitale Medien herunterladen
\end{itemize}
\paragraph{Nutzer aus der Gruppe Lehrstuhlmitarbeiter kann ...}
\begin{itemize}
	\item /M26/ eine Übersicht über die aktuell ausgeliehenen Medien erhalten
	\item /M27/ ein neues Medium hinzufügen
	\item /M28/ Daten von Medien ändern
	\item /M29/ einen neuen Nutzer zur Gruppe Studenten hinzufügen
	\item /M30/ Daten von Nutzern aus der Gruppe Studenten ändern
\end{itemize}
\paragraph{Nutzer aus der Gruppe Administratoren kann ...}
\begin{itemize}
	\item /M31/ einen neuen Nutzer zur Gruppe Studenten, Lehrstuhlmitarbeiter oder Administratoren hinzufügen
	\item /M32/ Daten von allen Nutzern ändern
	\item /M33/ Rechte von Gruppen ändern
\end{itemize}
\subsection{Wunschkriterien}
\begin{itemize}
	\item /W01/ Einführung einer neuen Medium-Kategorie durch Lehrstuhlmitarbeiter
	\item /W02/ Explizite Änderung vom Ausleihdauer für jede Ausleihe
	\item /W03/ System für das Zurücksetzen vom Passwort
	\item /W04/ Skalierbarkeit vom System, um es leicht um weitere Lehrstühle erweitern zu können
\end{itemize}
\subsection{Abgrenzungskriterien}
\begin{itemize}
	\item /A01/ keine Unterstützung für Touchscreens
	\item /A02/ kein großer Fokus auf das Sicherheitssystem des Systems, nicht immun gegen Hacker-Angriffe
\end{itemize}
\pagebreak

\section{Produkteinsatz}
\subsection{Anwendungsbereiche}
Der Dienst wird für die Benutzung der Lehrstuhlbibliothek für die Softwaretechnologie verwendet.  Das System soll mehrere Optionen anbieten, wie z.B Suche nach den Medien, Ausleihe, Verlängern, Monitoring.Das System verwaltet sowie physische als auch digitale Medien.
\subsection{Zielgruppen}
Personengruppen, die Lehrstuhlmitarbeiter und Studenten sind, können die Bibliothek benutzen. Gäste können nur nach den Medien suchen. Die Nutzer, die der Gruppe "Lehrstuhlmitarbeiter" gehören, können auch Administrator-Rechte haben.
\subsection{Betriebsbedingungen}
Dieses System soll folgende Bediengungen erfüllen:
\begin{itemize}
\item Betriebsdauer: täglich, 24 Stunden
\item Wartungsfrei
\item Die Sicherung der Datenbank muss manuell vom Administrator durchgefuhrt werden
\item Das System soll auf allen Betriebssystemen operabel sein/Das System soll von jedem Rechnersystem am Lehrstuhl erreichbar sei
\end{itemize}
\pagebreak

\section{Technische Produktumgebung}
\subsection{Software}
\subsection{Hardware}
\subsection{Orgware}
\subsection{Produktschnittstellen}
\pagebreak

\section{Produktfunktionen}

Die Software soll in der Lage sein, die Verwaltung einer Bibliothek durchzuführen. Hier werden die Punkte aus dem Lastenheft erneut aufgeführt und sofern weiter möglich näher erklärt.
\begin{itemize}
	\item /LF10/ das Ausleihen und Zurückgeben von physischen Medien unterschiedlicher Arten ist möglich

		Das Ausleihen und Zurückgeben erfolgt über die Eingabe der Medien-ID oder das Auswählen des Mediums in der Suche, bzw. bei der Rückgabe durch Auswahl im Account.
		Bei digitalen Medien ist keine Zurückgabe notwendig, hier wird nur der Zugriff in der Ausleihhistorie vermerkt.
	\item /LF20/ digitale Medien sind zum Download/zur Einsicht verfügbar

	Digitale Medien sind über die Suche aufrufbar, dem Nutzer soll dann die Wahl bleiben, ob er das Medium vollständig herunterladen oder sich nur kurzfristig anzeigen lassen möchte
	\item /LF30/ Mahnungen und Rückgabeerinnerungen werden automatisch versendet

	Die Bestätigung des Versands dieser Erinnerungen soll in der Monitoringübersicht einsehbar sein.

	\item /LF40/ Lehrstuhlmitarbeiter können eine Übersicht über die aktuell ausgeliehenen Medien erhalten

	\item /LF50/ das Ausleihen ist nur Studenten und Mitarbeitern möglich

	Dies wird über die Gruppenrechte kontrolliert, Gastnutzer erhalten die notwendigen Rechte nicht.
	\item /LF60/ Ausleihen können verlängert werden, wenn das Medium nicht vorbestellt ist
	\item /LF70/ eine Mediensuche ist über sämtliche Metadaten und den Inhalt möglich

	Die Metadaten sind unter \textit{5 Produktdaten /PD11/} aufgeführt
\end{itemize}
\pagebreak

\section{Produktdaten}
\begin{itemize}
	\item /LD10/ die Bibliothek wird auf längere Sicht <50.000 Medien umfassen
	\item /PD11/ die Metadaten eine Mediums sind Name, Veröffentlichungsdatum, Autor, Kategorie/Genre und Verfügbarkeit sowie optional der Verlag
\end{itemize}

Die innerhalb eines Jahres voraussichtlich abzuspeichernden Daten sind:
\begin{itemize}
	\item /LD20/ Nutzer- und Gruppeninformationen mit <1.000 Einträgen
	\item /PD21/ Nutzerinformationen umfassen Namen, Logininformationen, Mitarbeiter-/Matrikelnummer, Gruppenzugehörigkeit und aktuelle Ausleihen
	\item /PD22/ Gruppendaten umfassen den Namen, Rechte zur Ausleihe und Verwaltung und die mögliche Ausleihdauer
	\item /LD30/ Ausleihe- und Monitoringdaten mit <50.000 Einträgen
	\item /PD31/ die Ausleihdaten beinhalten die Ausleihhistorie aller Medien und der zugehörigen Nutzer, sowie alle aktuellen Ausleihen
	\item /PD32/ die Monitoringdaten beinhalten die Informationen zu gezahlten und ausstehenden Mahngebühren
\end{itemize}
\pagebreak

\section{Produktleistungen}
Solange die unter "5 Produktdaten" genannten Datenmengen nicht überschritten werden, gelten die Leistungsansprüche aus dem Lastenheft. Ansonsten steigen die zu akzeptierenden Leistungseinbußen proportional zu der zusätzlichen Datenmenge.
\begin{itemize}
	\item /LL10/ Anfragezeiten von <2sec für Mediensuchen
	\item /LL20/ Anfragezeiten von <1sec für Nutzer- und Gruppensuchen
	\item /LL30/ Anfragezeiten von <2sec für Ausleihe- und Monitoringdaten
	\item /LL40/ Zeitliche Fristen sind auf 24-Stunden-Basis beschränkt, also reicht eine zeitliche Genauigkeit von 1 min
	\item /LL50/ Abmahngebühren müssen in korrekten Euro-Werten angegeben werden
\end{itemize}
\pagebreak

\section{Benutzeroberfläche}
\pagebreak

\section{Qualitätsanforderungen}
\newcommand{\xmark}{\ding{55}}%
\begin{table}[h]
	\centering
\begin{tabular}{ || c | c | c | c | c || }
	\hline
	Systemqualität & sehr gut & gut & normal & nicht relevant  \\ \hline
	Funktionalität &  & \xmark &  &  \\ \hline
	Zuverlässigkeit &  & \xmark &  &  \\ \hline
	Benutzbarkeit &  \xmark & &  &  \\ \hline
	Effizienz &  &  & \xmark  &  \\ \hline
	Wartbarkeit &  &  & \xmark &  \\ \hline
	Portabilität &  &  &  & \xmark \\
	\hline
\end{tabular}
\caption{Qualitätszielbestimmung}
\label{table:qualitätsanforderungen}
\end{table}
\begin{itemize}
	\item /LQB10/ Qualitätsanforderung zur Benutzbarkeit des Systems
	\item /LQE10/ Qualitätsanforderung zur Effizienz des Systems
	\item /LQF10/ Qualitätsanforderung zur Funktionalität des Systems
	\item /LQP10/ Qualitätsanforderung zur Portabilität des Systems
	\item /LQW10/ Qualitätsanforderung zur Wartbarkeit des Systems
	\item /LQZ10/ Qualitätsanforderung zur Zuverlässigkeit des Systems
\end{itemize}
\pagebreak

\section{Globale Testszenarien/Akzeptanztestfälle}
\begin{table}[h]
	\begin{tabular}{p{1.2cm}|p{6.5cm}|p{5.5cm}}
		\textbf{Id} & \textbf{Beschreibung} & \textbf{Erwartetes Ergebnis}\\
		\hline
		/T01/ & Zugang auf die Webseite außerhalb des Lehrstuhls & Seite nicht verfügbar\\
		\hline
		/T02/ & Einloggen mit unkorrekten Daten & Einloggen fehlgeschlagen\\
		\hline
		/T03/ & Einloggen mit korrekten Daten & Einloggen erfolgreich\\
		\hline
		/T04/ & Ausloggen & Ausloggen erfolgreich\\
		\hline
		/T05/ & Ein paar Testpersonen gleichzeitig greifen auf die Seite zu & Kein Absturz\\
		\hline
		/T06/ & Hinzufügen vom Student zur Gruppe Studenten durch Lehrstuhlmitarbeiter & Der Student wird Nutzer\\
		\hline
		/T07/ & Löschen vom letzten Administrator durch diesen Administrator & Fehlermeldung\\
		\hline
		/T08/ & Hinzufügen vom Lehrstuhlmitarbeiter zur Gruppe Administratoren durch Administrator & Der Lehrstuhlmitarbeiter wird Administrator\\
		\hline
		/T09/ & Löschen vom einen Administrator durch Administrator & der Nutzer wird zur Gruppe Lehrstuhlmitarbeiter gehören\\
		\hline
		/T10/ & Hinzufügen vom Student zur Gruppe Administratoren & Fehlermeldung\\
		/T11/ & Hinzufügen vom Lehrstuhlmitarbeiter durch Administrator zur Gruppe Studenten & Fehlermeldung\\
		\hline
		/T12/ & Hinzufügen eines Buches vom Titel  'Buch1'  (ohne andere Metadaten) zum System & Fehlermeldung bei der Validierung\\
		\hline
		/T13/ & Hinzufügen eines Buches vom Titel  'Buch1'  (mit anderen Metadaten) zum System & Ein Buch vom Titel  'Buch1'  wurde zum Katalog hinzugefügt\\
    \end{tabular}
\end{table}

\begin{table}
	\begin{tabular}{p{1.2cm}|p{6.5cm}|p{5.5cm}}
		\hline
		/T14/ & Hinzufügen eines E-Books vom Titel  'Buch2'  und Inhalt  '...lorem ipsum'  (mit anderen Metadaten) zum System & Ein E-Book wurde zum Katalog hinzugefügt\\
		\hline
		/T15/ & Suchen über das Wort  'Buch'  & Zwei Einträge gefunden\\
		/T16/ & Suchen über das Wort  'Buch1'  & Ein Eintrag gefunden\\
		\hline
		/T17/ & Suchen über das Wort  'lorem'  & Ein Eintrag gefunden\\
		\hline
		/T18/ & Anschauen vom E-Book  'Buch2'  & Anschauen war möglich\\
		\hline
		/T19/ & Herunterladen vom E-Book  'Buch2'  & Herunterladen wurde erfolgreich abgeschlossen\\
		\hline
		/T20/ & Ausleihen vom Buch  'Buch1'  & Ausleihe erfolgreich abgeschlossen\\
		\hline
		/T21/ & Änderung von der Ausleihedauer für die Gruppe Studenten auf 3 Tage durch Administrator & Erfolgreiche Änderung\\
		\hline
		/T22/ & Drei Tage vor der erwarteten Abgabe für ein Medium & Versand einer Rückgabeerinnerung zum Student über 'Buch1' \\
		\hline
		/T23/ & Änderung vom Titel vom  'Buch1'  zum  'Buch3'  & Änderung erfolgreich\\
		\hline
		/T24/ & Abrufen von einer Übersicht über die aktuell ausgeliehenen Medien erhalten & korrekte Daten angezeigt ( 'Buch 3'  ist ausgeliehen)\\
		\hline
		/T25/ & Vormerken vom Buch 'Buch3' durch anderen Nutzer & Das Buch wurde vorgemerkt\\
		\hline
		/T26/ & Verlängern vom Buch 'Buch3' & Verlängerung nicht möglich, weil es schon vorgemerkt wurde\\
		\hline
		/T27/ & das Ausleihdatum ist vorbei & Versand einer Mahnung zum Student über  'Buch1'  mit der Strafangaben in Euros \\
		\hline
		/T28/ & Rückgabe vom Buch  'Buch1'  & Rückgabe erfolgreich abgeschlossen\\
		\hline
		/T29/ & Abrufen von einer Übersicht über die aktuell ausgeliehenen Medien erhalten & korrekte Daten angezeigt (leer)\\
		\hline
		/T30/ & [mehrere Medien hinzugefügt] Schicken einer Anfrage für ein Medium & Anfragezeit von \textless 2sec\\
		\hline
		/T31/ & [mehrere Nutzer hinzugefügt] Schicken einer Anfrage für ein Nutzer & Anfragezeit von \textless 1sec\\
		\hline
		/T32/ & Schicken einer Anfrage für eine Gruppe & Anfragezeit von \textless 1sec\\
		\hline
		/T33/ & [mehrere Ausleihen durchgeführt] Schicken einer Anfrage für eine Ausleihe & Anfragezeit von \textless 2sec\\
    \end{tabular}
\end{table}
\pagebreak

\section{Glossar}
\pagebreak

\end{document}
